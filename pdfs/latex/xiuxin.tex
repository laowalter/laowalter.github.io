\documentclass{article}
\usepackage{geometry}
\geometry{b5paper,left=2.5cm,right=2.5cm,top=3cm,bottom=1cm}

\usepackage{xeCJK} % 这个包可以指定中文字体。
\newcommand{\hei}{\CJKfamily{hei}}
\setCJKfamilyfont{hei}{Adobe Heiti Std}
\newcommand{\song}{Adobe Song Std}
\setCJKfamilyfont{\song}{Adobe Heiti Std}
\newcommand{\kai}{Adobe Kaiti Std R}
\setCJKfamilyfont{\song}{Adobe Heiti Std}
\usepackage{fontspec}
\setCJKmainfont{\song} 

\begin{document}
\thispagestyle{empty}
\begin{center}
    \hei{\Huge 修心十七条}
\end{center}
\hfill

\large\selectfont
\begin{enumerate}
\item 修心沒有終點,了願了,要再來,來了還要再來,吃苦了,苦而已,唯有隨喜自在才是真自在。
\item 人生不能以感覺過一生,要用心過一生。
\item 心善一切善,心每一切美,人能自覺,才能放下。
\item 生命要有深度、廣度,還要有名度,看得清、灑脫,對待任何事情能夠臨危不懼,對待這樣的人生還能無怨無悔。
\item 以自覺的心成長,以自信的心尊重,以自愛的心珍視,以自性的心涵養。
\item 在受苦的時候,在逆境的時候,心要做平常心,就不覺得苦,就不覺得是在逆境狀態。
\item (N/A)
\item 把握當下即是悟。
\item 做一切決定衹要利益衆生,都是最好的。
\item 衹要保持一顆赤誠之心,相信自己就是相信別人,相信老天就是相信自己。
\item 修心要先從心做起,所謂相由心生,不會被外界影響。
\item 任何事情要提得起放得下,解決之後都要化為精神,不要化爲負擔。
\item 能看透,能看清,能灑脫,碰到事情臨危不懼。
\item 當你整成的時候,你不必須得到真誠,當真誠的時候不用祈求天,自然有感應,因爲當下天就是心。
\item 遇到每一個瓶頸都是成長的時候,不要看表面的情況,做人要深入,做人不能表面化,你表面化了,別人也會對你表面化。做人要成長,若不是30年來某先生不是從一個瓶頸一個瓶頸中走過來,就沒有他的成長。
\item 修的好也要讲的好,口心行合一,一致同步化。
\item 只要赤诚就能感天,就能灵验。
\end{enumerate}
\end{document}
